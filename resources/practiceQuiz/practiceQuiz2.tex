\documentclass[addpoints,12pt]{exam}
\usepackage{color}
\usepackage{amsmath}

\begin{document}
\title{Quiz 2 Practice Questions}
\author{}
\date{}
\maketitle

\newcommand{\hint}[1]{\\{\bf Hint:} {\em {#1}}}



\begin{center}
{\Large \textbf{Algorithmic Questions}}
\end{center}


\fullwidth{\Large \textbf{Topic : Trees}}
\begin{questions}
\qformat{[BST\thequestion]}
\question
Suppose you are given a completely balanced binary search tree (a completely balanced tree 
means that each path from root to leaf is exactly the same). What is the time required to find the median
of the elements of such a tree ?

\question
Suppose you are given a completely balanced binary search tree (a completely balanced tree 
means that each path from root to leaf is exactly the same). What is the time required to find the max 
of the elements of such a tree ?

\question
Suppose some one gave you a binary tree and claimed that it is a BST. 
Design and analyze an efficient algorithm to verify if it is indeed the case.

\question
Given a BST $T$ and two integers $a$ and $b$, print all elements in $T$ between $a$ and $b$.

\question
Suppose you are given two trees $T_1$ and $T_2$. Design an efficient algorithm to determine if they contain the same elements.

\question
Design an algorithm to convert a {\em sorted} array to a {\em balanced} binary tree.

\end{questions}

\fullwidth{\Large \textbf{Red-Black Trees}}
\begin{questions}
\qformat{[RBT\thequestion]}

\question
Insert the following numbers into an initially empty red-black tree: $8, 2, 4, 7, 5, 3, 1, 6$. 
(To solve this problem correctly, you will need to refer to the text-book for details about Red-Black trees in 
addition to the main ideas discussed in class).

\question
Design a ``best-case'' red-black tree with 10 nodes, i.e., a red-black tree with the longest 
possible path from the root to a leaf.

\question
Design a ``worst-case'' red-black tree with 10 nodes, i.e., a red-black tree with the longest 
possible path from the root to a leaf.

\question
What is minimum possible number of nodes in a red-black tree which contains two black nodes 
from every root-to-leaf path? 

\question
Suppose we define a red-black tree where along each path the number of black nodes is the 
same, and there cannot be more than three consecutive red nodes (but there may be two 
consecutive red nodes).
\begin{parts}
\part
What is the ratio of the longest possible path length to the shortest possible path length in this tree? 
\part
For a tree that has b black nodes along each path, what ratio of the maximum possible 
number of nodes to the minimum possible number of nodes in the tree? 
\end{parts}

\end{questions}

\fullwidth{\Large \textbf{Binary Heaps}}
\begin{questions}
\qformat{[Heap\thequestion]}

\question
Given a heap and a number $k$, design an efficient algorithm that outputs the top-$k$ largest 
element from the heap. What is the running time of the algorithm? Can you design an algorithm
that runs faster than $O(k \log n)$? 

\question
Instead of binary heaps, suppose you had to implement ternary heaps (i.e., where each node has 
up to three children). Explain how you would implement such heaps using arrays, and how you 
can determine child and parent pointers. What are the advantages/disadvantages of ternary 
heaps over binary heaps?

\question
Can you use a binary search tree to simulate heap operations? What are the 
advantages/disadvantages of doing so?

\question
Design an algorithm to convert a given binary search tree into a heap efficiently. What is the 
running time of your algorithm

\question
Assume a heap is arranged such that the largest element is on the top. Suppose you are given 
two heaps $S$ and $T$, each with $m$ and $n$ elements respectively. What is the running time of an 
efficient algorithm to find the $10^{th}$ largest element of $S \cup T$?

\end{questions}


\fullwidth{\Large \textbf{Union-Find}}
\begin{questions}
\qformat{[UF\thequestion]}

\question
Can you use any of the previously studied data structures (e.g. heaps, red-black trees) for the Union-Find problem? Explain your answer.

\question
What is the the worst-case performance of the Union-Find data structure we discussed (with union by rank and path-compression)?

\question
Suppose we are working with a Union-Find data structure as described in class (assume no path
compression is applied). Suppose we consider a set of 10 items that are eventually joined into a single
set via 9 union operations
\begin{parts}
\part
Describe the sequence of union operations that will result in the root having the largest number of children. What is the final degree of the root in this case?
\part
Can you have a sequence of union operations that will result in the root having three children? Explain why or why not.
\end{parts}

\question
Suppose we are working with a Union-Find data structure as described in class (this time path compression is allowed for Find operations). 
Node $z$ is at a distance of $4$ from the root $r$ of its tree (i.e.  $4$ edges away). We now perform a Find on $z$.
\begin{parts}
\part
By how much will the degree of $r$ change? Explain your answer.
\part
By how much will the degree of $z$ change? Explain your answer
\end{parts}

\question
Recall the definition of the ``iterated logarithm function" $\log^{*}(n)$. 
\begin{parts}
\part
Similarly, try and define the ``iterated square root function'' $sqrt^{*}(n)$.
\part
What is the value of $sqrt^{*}(2^32)$?
\end{parts}

\question
Determine the incorrect statement below concerning the Union-Find data structure we
discussed in class (with union by rank and path-compression).
\begin{parts}
\part The union operation sometimes may not increase the height of the resultant tree
\part The union operation can at most increase the height of the resultant tree by one
\part The union operation always increases the height of the resultant tree by one 
\part The find operation sometimes may not increase the degree of the root of the resultant tree
\part The find operation requires two traversals from node to root
\end{parts}

\end{questions}

\end{document}

